\documentclass[12pt,a4paper]{article}
\usepackage[left=2cm,right=2cm,top=2cm,bottom=2cm]{geometry}
\usepackage[utf8]{inputenc}
\usepackage{amsmath}
\usepackage{siunitx}
\usepackage{amsfonts}
\usepackage{amssymb}
\usepackage{graphicx}
\usepackage[ngerman]{babel}
\usepackage{lmodern}
\begin{document}
\section{Materialliste}
Der Versuchsaufbau besteht aus:
\begin{itemize}
	\item Koordinatenunterlage
	\item nichtmagnetischer Befestigung
	\item Schraubzwingen
	\item Leybold Steckbrett
	\item zwei Laser
	\item ein Helmholtzspulenpaar
	\begin{itemize}
		\item Innenradius: \SI{15.0(1)}{\centi m}
		\item Außenradius: \SI{16.1(1)}{\centi m}
		\item Achsiale Breite: \SI{2.0(1)}{\centi m}
		\item Windungszahl: \SI{130}{}
	\end{itemize}
	\item zwei optische Bänke
\end{itemize}
Für die Messungen wurden die folgenden Messgeräte verwendet:
\begin{itemize}
\item Cassy System
\item drei Hall-Sensoren
\item Spannungsmessgerät
\item Strommessgerät
\item Temperatursensor
\item Winkelmesser
\item Lineal
\end{itemize}
Die Hallsonden wurden während des Versuchs mit einer 7mA Stromquelle 
sowie die Spulen mit einer 4A Konstantstromquelle betrieben.

\end{document}
