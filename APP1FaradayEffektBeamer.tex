\documentclass[11pt,a4paper]{beamer}
\usepackage[utf8]{inputenc}
\usepackage{amsmath}
\usepackage{amsfonts}
\usepackage{amssymb}
\usepackage{graphicx}
\author{Henrik Jürgens}
\title{\textbf{Faraday Effekt}\\ weekly report group B}
\begin{document}
\maketitle
\frame{\tableofcontents}
% \frame{\tableofcontents[currentsection]}
 \section{Theory for the Faraday-Effect}
\begin{frame} %%Eine Folie
\subsection{What is the Faraday-Effect}
  \frametitle{What is the Faraday-Effect} %%Folientitel
  \begin{enumerate}
  \item[$\blacktriangleright$]
  	The Faraday-Effect describes an interaction between light, as an electromagnetic wave, and a given magnetic field \textbf{B} in a nonconductive transparent medium
  \newline
  \item[$\blacktriangleright$]
    It causes a rotation $\beta$ of the plane of polarization, which is proportional to the component of the magnetic field in the direction of propagation
  \end{enumerate}
\end{frame}
\begin{frame}
\subsection{Assumptions}
\frametitle{Assumptions}
\begin{enumerate}
\item[$\blacktriangleright$]
There are microscopic magnetical moments in every nontransparent and nonconductive medium in an atomic scale\newline\newline
 Classical: Electrons orbiting the nucleus of the atom\newline
 QM: The total angular momentum $\vec{\textbf{J}}$ of Electrons\newline
\item[$\blacktriangleright$]
We have a linear transverse electromagnetic wave (homogene Maxwell-Equations)
  
\end{enumerate}
\end{frame}
\begin{frame}
\subsection{Derivation of the rotation angle $\beta$}
\frametitle{Derivation of the rotation angle $\beta$}
\begin{itemize}
\item[$\blacktriangleright$]
Starting with the gyroscopic equation:
\begin{align}
\dot{\vec{\mu}} = \gamma\cdot\vec{\mu}\times\vec{\textbf{B}}
\label{Kreiselgleichung}
\end{align}
$\Rightarrow$ Precession of the magnetical moments
\newline \newline
QM : $\gamma = g\frac{q}{2m}$, $g$ the Lande-Faktor
(Classically: $g=1$)\newline \newline
\item[$\blacktriangleright$]
w.l.o.g. $\vec{\mu} \perp \vec{\textbf{B}}$, so we obtain the Larmor frequency
\begin{align}
\omega_L = \Delta E_{m_J}/ \hbar = g\frac{q}{2m}B \qquad (B = |\vec{\textbf{B}}|)
\label{Larmorfrequenz}
\end{align}
as the precession frequency of $\vec{\mu}$ from eq. (\ref{Kreiselgleichung})
\end{itemize}
\end{frame}
\begin{frame}
\begin{itemize}
\item[$\blacktriangleright$]
To describe the linear transverse electromagnetic wave we start with the dispersion relation:
\begin{align}
\omega = k'(\omega)\cdot c = k\cdot n(\omega)\cdot c
\label{Dispersionsrelation}
\end{align}
and obtain (after moving to the complex plane for simplicity)
 \begin{align}
\textbf{E} = E_0\cdot[e^{i( k'_l x - \omega t)}+e^{-i( k'_r x - \omega t )}] = \textbf{E}_{\text{l}} + \textbf{E}_{\text{r}}
\label{Lichtwelle_1}
 \end{align} 
where the light is propagating perpendicular to the complex plane (we will just look at the $\textbf{E}$-field in the following)\newline
\item[$\blacktriangleright$] So we obtain a superposition of a right- and left-handed circular wave with total amplitude $2E_0$
\end{itemize}
\end{frame}
\begin{frame}
\begin{itemize}
\item[$\blacktriangleright$] We have to move in the system of the electrons, to get a picture of the wave properties
\item[$\blacktriangleright$] As we see in eq. (\ref{Kreiselgleichung}) the electons precess with the magnetical moments counterclockwise around $\vec{\textbf{B}}$\newline\newline
$\Rightarrow$ In the \textbf{perspective of the electron}:\newline $k'_l = k'(\omega + \omega_L)$ for a left-handed circular wave and $k'_r = k'(\omega - \omega_L)$ for a right-handed circular wave
\item[$\blacktriangleright$] Inserted in (\ref{Lichtwelle_1}) we obtain:
\begin{align}
\textbf{E} = E_0\cdot[e^{i(k'(\omega + \omega_L) x - \omega t)}+e^{-i(k'(\omega - \omega_L) x - \omega t)}]
\label{Lichtwelle_2}
\end{align}
\end{itemize}
\end{frame}
\begin{frame}
\begin{itemize}
\item[$\blacktriangleright$] Assuming $\omega_L << \omega$ $\Rightarrow$
\begin{align*}
k'_{\pm} := k'(\omega \pm \omega_L) \approx k'(\omega) \pm \frac{\text{d} k'}{\text{d} \omega}(\omega)\cdot\omega_L := k' \pm \frac{\text{d}k'}{\text{d}\omega}\omega_L
\end{align*}
\item combined with (\ref{Lichtwelle_2}) we get:
\begin{align}
\textbf{E}& = E_0\cdot[e^{i((k'+ \frac{\text{d} k'}{\text{d} \omega}\omega_L) x - \omega t)}+e^{-i((k' - \frac{\text{d} k'}{\text{d} \omega}\omega_L) x - \omega t)}] \notag \\
& = E_0\cdot[e^{i\frac{\text{d} k'}{\text{d} \omega}\omega_L \cdot x}(e^{i(k' x - \omega t)}+e^{-i(k' x - \omega t)})]
\label{Lichtwelle_3}
\end{align}
\end{itemize}
\end{frame}
\begin{frame}
\begin{itemize}
\item[$\blacktriangleright$] As a consequence of eq. (\ref{Lichtwelle_3}), with $\Delta x$ being the length of the interval the light has travelled in the medium, we obtain:
\begin{align}
\beta = \frac{\text{d} k'}{\text{d} \omega}\omega_L\cdot \Delta x = \frac{\text{d} k'}{\text{d} \omega}\frac{e}{2m_e}B\cdot \Delta x
\label{Drehwinkel_1}
\end{align}
(by using $g = 1$ und $ q = e $ in (\ref{Larmorfrequenz}) for simplicity)
\item[$\blacktriangleright$] With (\ref{Dispersionsrelation}) we see
\begin{align}
\frac{\text{d} k'}{\text{d} \omega} = \frac{2\pi}{\lambda}\cdot\frac{\text{d} n}{\text{d} \lambda}\frac{\text{d} \lambda}{\text{d} \omega} = \frac{2\pi}{\lambda}\cdot \frac{\text{d} n}{\text{d} \lambda} \cdot \frac{-\lambda^2}{2\pi c} = -\frac{\lambda}{c} \cdot \frac{\text{d} n}{\text{d} \lambda}
\end{align}
and get with eq. (\ref{Drehwinkel_1})
\begin{align}
\beta = -\frac{\lambda}{c} \frac{\text{d} n}{\text{d} \lambda}\frac{e}{2m_e}\cdot\Delta x\cdot B = \Delta x \cdot V \cdot B
\label{Drehwinkel_2}
\end{align}
where $V$ is the Verdet constant of the medium
\end{itemize}
\end{frame}
\begin{frame}
\begin{itemize}
\item[$\blacktriangleright$] For non homogeneous \textbf{B}-field we let $\Delta x \rightarrow 0$ to get the more general formula for $\beta$
\begin{align}
\beta_{l_1\rightarrow l_2} = V \cdot \int_{l_1}^{l_2} B dx
\end{align}
\end{itemize}
\end{frame}
\begin{frame}
\subsection{Comments to characteristic features of the Faraday-Effect}
\frametitle{Comments to characteristic features of the Faraday-Effect}
\begin{itemize}
\item[$\blacktriangleright$] With a strong \textbf{B}-Field we know $g > 1$, so the precession frequency will be bigger than the classical Larmor frequency
\item[$\blacktriangleright$] The Verdet constant dependens on the temperature of the medium ($\frac{\text{d} n}{\text{d} \lambda}(T)$ in the area of small variations negligible)
\item[$\blacktriangleright$] $\beta$ ist independent on the direction of propagation of the light.
Because left/right-handed circular polarized light changes to right/left-handed circular polarized light by changing the direction of propagation and
we see, that eq. (\ref{Lichtwelle_1}) is invariant under the transformation 
\begin{align}
\begin{pmatrix}
k_r \\
k_l \\
\omega
\end{pmatrix} \rightarrow 
\begin{pmatrix}
-k_l\\
-k_r\\
-\omega
\end{pmatrix}
\end{align}
As a consequence, we get $\beta = \beta'$ \newline(\textbf{the specific difference from the Faraday-Effect to other double refraction phenomena})
\end{itemize}
\end{frame}
\end{document}
