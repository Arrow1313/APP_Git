\documentclass[12pt,a4paper]{article}
\usepackage[left=2cm,right=2cm,top=2cm,bottom=2cm]{geometry}
\usepackage[utf8]{inputenc}
\usepackage{amsmath}
\usepackage{amsfonts}
\usepackage{amssymb}
\usepackage{graphicx}
\usepackage[ngerman]{babel}
\usepackage{lmodern}
\usepackage{siunitx}
\begin{document}
\section{Versuchsdurchführung}
Bevor eine Messung gestartet werden kann, sind verschiedene Aspekte zu beachten:\\
Die Offsets der Skalen der optischen Bänke zur Koordinatenunterlage müssen bestimmt, bzw. von dem Vorgänger übernommen werden.\\ Die Laser können dann mit einem Winkelmaß und einem Lineal, welches im rechten Winkel zur Tischebene aufgestellt wird, anhand der Skala der Koordinatenunterlage zueinander orthogonal und in gleicher Höhe in der Tischebene ausgerichtet werden.\\ Davor sollte stets Überprüft werden, ob das Laserlicht in der Tischebene liegt, da es möglich ist, dass die Laser ihre Ausrichtung bei Stößen gegen den Aufbau ändern (Höhendifferenz in 1 Meter Abstand zum Laser $\leq$ \SI{1}{\milli \meter}). Während der Messung werden die Laser auf den optischen Bänken parallel verschoben und mit maximal \SI{20}{\milli \ampere} versorgt, damit sie nicht überhitzen.\\ Die Spulen werden an eine \SI{4}{\ampere} und die Hallsonden an eine \SI{7}{\milli \ampere} Konstantstromquelle angeschlossen. Zur Kontrolle der Ströme werden zusätzlich zwei Digitalmultimeter in Reihe geschaltet. Die Ausrichtung des Hallsensors erfolgt vor jedem Messwert mithilfe der bereits orthogonal zueinander ausgerichteten Laser. An der Halterung des Hallsensors wurde dafür ein Punkt markiert, welcher von einem Teil des Laserstrahls angeleuchtet werden muss, während der andere Teil die Sensorspitze erfasst. Die Messwerte werden dann nach einem vorgegebenen Raster nacheinander abgefahren\\ Ziel der Messungen ist es die Homogenität in der Mitte der Spule zu erfassen und mit der Simulation zu vergleichen. Die Änderung der Homogenität bei doppeltem Spulenabstand kann bei genügend Zeit ebenfalls ausgemessen werden. 
\end{document}