\documentclass[10pt,a4paper]{beamer}
\usepackage[utf8]{inputenc}
\usepackage{amsmath}
\usepackage{amsfonts}
\usepackage{amssymb}
\usepackage{graphicx}
\author{Henrik Jürgens}
\title{\textbf{Faraday Effekt}\\ weekly report group B}
\begin{document}
%\maketitle
%\frame{\tableofcontents}
% \frame{\tableofcontents[currentsection]}
 \section{Theory of the Faraday-Effect}
\begin{frame} %%Eine Folie
\subsection{Assumptions}
\frametitle{\textbf{Theory of the Faraday-Effect:} \\ Assumptions}
\pause
\begin{enumerate}
\item[$\blacktriangleright$]
There are microscopic magnetic moments in every transparent and nonconductive medium in an atomic scale\newline\newline\pause
 Classical: Electrons orbiting the nucleus of the atom\newline
 QM: The total angular momentum $\vec{\textbf{J}}$ of Electrons\newline\pause
\item[$\blacktriangleright$]
Linear polarized light is a transverse electromagnetic wave (homogene Maxwell-Equations)
  
\end{enumerate}
\end{frame}
\begin{frame}
\subsection{Derivation of the rotation angle $\beta$}
\frametitle{Derivation of the rotation angle $\beta$}
\begin{itemize}
\item[$\blacktriangleright$]
Gyroscopic equation:
\begin{align}
\dot{\vec{\mu}} = \gamma\cdot\vec{\mu}\times\vec{\textbf{B}}
\label{Kreiselgleichung}
\end{align}\pause
$\Rightarrow$ Precession of the magnetical moments:
\begin{align}
\omega_L = \Delta E_{m_J}/ \hbar = g\frac{q}{2m}B \qquad (B = |\vec{\textbf{B}}|)
\label{Larmorfrequenz}
\end{align}\pause
$\omega_L \equiv$ precession frequency of $\vec{\mu}$ ('\textbf{Larmor frequency}')
\newline
\begin{small}(QM : $\gamma = g\frac{q}{2m}$, $g$ the Lande-Faktor;
Classically: $g=1$)
\end{small}\newline\pause
\item[$\blacktriangleright$]
Dispersion relation:
\begin{align}
\omega = k'(\omega)\cdot c = k\cdot n(\omega)\cdot c
\label{Dispersionsrelation}
\end{align}\pause
\item[$\blacktriangleright$]
In the complex plane \small{(light is propagating perpendicular to it)}:
 \begin{align}
\textbf{E} = E_0\cdot[e^{i( k'_l x - \omega t)}+e^{-i( k'_r x - \omega t )}] = \textbf{E}_{\text{l}} + \textbf{E}_{\text{r}}
\label{Lichtwelle_1}
 \end{align}
\end{itemize}
\end{frame}
\begin{frame}
\begin{itemize}
\item[$\blacktriangleright$] \textbf{perspective of the electron}:\newline\pause
\begin{small}\noindent electons precessing counterclockwise around $\vec{\textbf{B}}$ with the magnetical moments: 
\end{small}\newline
$k'_l = k'(\omega + \omega_L)$ for a left-handed circular wave and \newline
$k'_r = k'(\omega - \omega_L)$ for a right-handed circular wave\newline\pause
\item[$\blacktriangleright$] Assuming $\omega_L << \omega$
\begin{align*}
\Rightarrow k'_{\pm} := k'(\omega \pm \omega_L) \approx k'(\omega) \pm \frac{\text{d} k'}{\text{d} \omega}(\omega)\cdot\omega_L := k' \pm \frac{\text{d}k'}{\text{d}\omega}\omega_L
\end{align*}\pause
\item combined with (\ref{Lichtwelle_1}) one obtains:
\begin{align*}
\textbf{E} = E_0\cdot[e^{i\frac{\text{d} k'}{\text{d} \omega}\omega_L \cdot x}(e^{i(k' x - \omega t)}+e^{-i(k' x - \omega t)})]\\
\uncover<5->{\Rightarrow \beta = \frac{\text{d} k'}{\text{d} \omega}\omega_L\cdot \Delta x = \frac{\text{d} k'}{\text{d} \omega}\frac{e}{2m_e}B\cdot \Delta x}
%\label{Lichtwelle_3}
%\label{Drehwinkel_1}
\end{align*}\pause
\begin{small}($\Delta x \equiv$ length of the interval the light has travelled in the medium\\
 $g = 1$ und $ q = e $ in (\ref{Larmorfrequenz}) for simplicity)
\end{small}
\end{itemize}
\end{frame}
\begin{frame}
\begin{itemize}
\item[$\blacktriangleright$] Using the chainrule and the dispersion relation:
\begin{align}
\frac{\text{d} k'}{\text{d} \omega} = \frac{2\pi}{\lambda}\cdot\frac{\text{d} n}{\text{d} \lambda}\frac{\text{d} \lambda}{\text{d} \omega} = \frac{2\pi}{\lambda}\cdot \frac{\text{d} n}{\text{d} \lambda} \cdot \frac{-\lambda^2}{2\pi c} = -\frac{\lambda}{c} \cdot \frac{\text{d} n}{\text{d} \lambda}
\end{align}\pause
\begin{align}
\Rightarrow \beta = -\frac{\lambda}{c} \frac{\text{d} n}{\text{d} \lambda}\frac{e}{2m_e}\cdot\Delta x\cdot B = \Delta x \cdot V \cdot B
\label{Drehwinkel_2}
\end{align}
where $V$ is the Verdet constant of the medium\pause
\item[$\blacktriangleright$] For non homogeneous \textbf{B}-field we let $\Delta x \rightarrow 0$ to get the more general formula for $\beta$
\begin{align}
\beta_{l_1\rightarrow l_2} = V \cdot \int_{l_1}^{l_2} B dx
\end{align}
\end{itemize}
\end{frame}
\begin{frame}
\subsection{Comments to characteristic features of the Faraday-Effect}
\frametitle{Comments to characteristic features of the Faraday-Effect}
\begin{itemize}
\item[$\blacktriangleright$] With a strong \textbf{B}-Field we know $g > 1$, so the precession frequency will be bigger than the classical Larmor frequency \newline \pause
\item[$\blacktriangleright$] The Verdet constant dependens on the temperature of the medium ($\frac{\text{d} n}{\text{d} \lambda}(T)$ in the area of small variations negligible) \newline \pause
\item[$\blacktriangleright$] $\beta$ is independent on the direction of propagation of the light.\newline \newline \pause
Left/right-handed circular polarized light changes to right/left-handed circular polarized light by changing the direction of propagation and
you see, that eq. (\ref{Lichtwelle_1}) is invariant under the transformation 
\begin{align}
\begin{pmatrix}
k_r \\
k_l \\
\omega
\end{pmatrix} \rightarrow 
\begin{pmatrix}
-k_l\\
-k_r\\
-\omega
\end{pmatrix}
\end{align}\pause
As a consequence, we obtain $\beta = \beta'$ \newline(\textbf{the specific difference from the Faraday-Effect to other double refraction phenomena})
\end{itemize}
\end{frame}
\end{document}
